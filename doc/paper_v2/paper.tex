%%%%%%%% ICML 2023 EXAMPLE LATEX SUBMISSION FILE %%%%%%%%%%%%%%%%%

\documentclass{article}

% Recommended, but optional, packages for figures and better typesetting:
\usepackage{microtype}
\usepackage{graphicx}
\usepackage{subfigure}
\usepackage{booktabs} % for professional tables

% hyperref makes hyperlinks in the resulting PDF.
% If your build breaks (sometimes temporarily if a hyperlink spans a page)
% please comment out the following usepackage line and replace
% \usepackage{icml2023} with \usepackage[nohyperref]{icml2023} above.
\usepackage{hyperref}


% Attempt to make hyperref and algorithmic work together better:
\newcommand{\theHalgorithm}{\arabic{algorithm}}

% Use the following line for the initial blind version submitted for review:
\usepackage[accepted]{icml2023}
% If accepted, instead use the following line for the camera-ready submission:
% \usepackage[accepted]{icml2023}

% For theorems and such
\usepackage{amsmath}
\usepackage{amssymb}
\usepackage{mathtools}
\usepackage{amsthm}
\usepackage{multirow}
% if you use cleveref..
\usepackage[capitalize,noabbrev, nameinlink]{cleveref}

%%%%%%%%%%%%%%%%%%%%%%%%%%%%%%%%
% THEOREMS
%%%%%%%%%%%%%%%%%%%%%%%%%%%%%%%%
\theoremstyle{plain}
\newtheorem{theorem}{Theorem}[section]
\newtheorem{proposition}[theorem]{Proposition}
\newtheorem{lemma}[theorem]{Lemma}
\newtheorem{corollary}[theorem]{Corollary}
\theoremstyle{definition}
\newtheorem{definition}[theorem]{Definition}
\newtheorem{assumption}[theorem]{Assumption}
\theoremstyle{remark}
\newtheorem{remark}[theorem]{Remark}

% Todonotes is useful during development; simply uncomment the next line
%    and comment out the line below the next line to turn off comments
%\usepackage[disable,textsize=tiny]{todonotes}
\usepackage[textsize=tiny]{todonotes}


% The \icmltitle you define below is probably too long as a header.
% Therefore, a short form for the running title is supplied here:
\icmltitlerunning{DRL for Energy Management}

\begin{document}

\twocolumn[
\icmltitle{Deep Reinforcement Learning for Energy Management in Residential Housing}

% It is OKAY to include author information, even for blind
% submissions: the style file will automatically remove it for you
% unless you've provided the [accepted] option to the icml2023
% package.

% List of affiliations: The first argument should be a (short)
% identifier you will use later to specify author affiliations
% Academic affiliations should list Department, University, City, Region, Country
% Industry affiliations should list Company, City, Region, Country

% You can specify symbols, otherwise they are numbered in order.
% Ideally, you should not use this facility. Affiliations will be numbered
% in order of appearance and this is the preferred way.
\icmlsetsymbol{equal}{*}

\begin{icmlauthorlist}
    \icmlauthor{Tim Walter}{yyy}
\end{icmlauthorlist}

\icmlaffiliation{yyy}{Department of Scientific Computing, Technical University of Munich, Munich, Germany}


\icmlcorrespondingauthor{Tim Walter}{tim.walter@tum.de}


% You may provide any keywords that you
% find helpful for describing your paper; these are used to populate
% the "keywords" metadata in the PDF but will not be shown in the document
\icmlkeywords{Machine Learning, ICML}

\vskip 0.3in
]

% this must go after the closing bracket ] following \twocolumn[ ...

% This command actually creates the footnote in the first column
% listing the affiliations and the copyright notice.
% The command takes one argument, which is text to display at the start of the footnote.
% The \icmlEqualContribution command is standard text for equal contribution.
% Remove it (just {}) if you do not need this facility.

%\printAffiliationsAndNotice{}  % leave blank if no need to mention equal contribution
\printAffiliationsAndNotice % otherwise use the standard text.

\begin{abstract}
    Buildings account for a significant portion of global energy consumption and emissions. This paper explores how smart energy management systems in single-family homes can potentially reduce emissions, focusing on homes equipped with photovoltaic systems, electric batteries, and system-controllable appliances. The aim is to optimize energy consumption and generation capacities to minimize emissions, manage load flexibility, and alleviate grid pressure from fluctuating renewable energy sources. The controller also enables the sale of cleanly generated electricity to the grid at a discounted emission premium. The control mechanisms involve deep reinforcement learning algorithms benchmarked against traditional thresholding models and the theoretical optimum.
\end{abstract}

\section{Introduction}\label{sec:introduction}
The operations of buildings account for 30\% of global final energy consumption and 26 \%
of global energy-related greenhouse-gas emissions \cite{IEA.06.01.2024}. This makes the optimization of energy consumption a major concern for environmental sustainability. This paper investigates the application of Deep Reinforcement Learning (DRL) to streamline energy usage within a single-family home, leveraging solar panels, electric batteries, heat pumps, and flexible demand mechanisms. The primary aim is to reduce the carbon footprint associated with housing operations, addressing a significant component of worldwide emissions.

Moreover,  aligning energy demand with periods of high renewable output, enabled by low carbon intensity electricity, not only aids environmental goals but also enhances grid stability. Dispatching reserves through batteries during periods of high carbon intensity fosters balanced power flow, contributing to a more resilient and efficient energy grid.

Furthermore, an evolving landscape in electricity contracts, featuring adaptive pricing structures, introduces a potential economic incentive for inhabitants. As the energy sector increasingly transitions towards renewable sources, synchronizing residential energy usage with favorable grid conditions not only supports ecological goals but may also translate into cost savings for residents.

\section{Related Work}\label{sec:related_work}
The application field of smart energy management encompasses a wide range of problems, such as heating and cooling \cite{Blum.2021}\cite{ThomasSchreiber.2020}, flexible demand response \cite{Jin.2021} and energy storage \cite{Nakabi.2021}. Moreover, the scale and level of detail of the tackled problems vary greatly, going from single buildings \cite{Blum.2021}, to microgrids \cite{Nakabi.2021} \cite{Castellanos.04.12.2022}, to eventually continent-wide electricity grids \cite{Horsch.2018}. The two most promising approaches to solve such problems are classical Model Predictive Control (MPC) \cite{Basantes.2023} and Reinforcement Learning (RL) \cite{Nakabi.2021} \cite{Jin.2021} \cite{ThomasSchreiber.2020}\cite{Zhu.2022}. Furthermore, there are also hybrid approaches that combine the two methods \cite{JavierArroyo.2022}.
\par
This paper is mainly inspired by \cite{Nakabi.2021}, who describe a microgrid environment with generation, storage, and consumers, where they optimize electricity pricing using various DRL algorithms. The inspiration is limited to the modelling of residential housing by the three major components Energy Storage System (ESS), Thermostatically Controlled Load (TCL), and Flexible Demand Response (FDR). Differentiation is given in the objective, which is to minimize the CO\textsubscript{2} equivalent emissions, and the setting, where the environment is a single household optimised from the perspective of the inhabitants instead of the electricity provider. Moreover, the dynamics of the ESS and in particular the FDR are original.

\section{Environment}\label{sec:environment}
The main contribution of this work is an environment modeling a single-family household, consisting of a Rooftop Solar Array (RSA), an ESS, a TCL, and a FDR. The environment is a mixture of replayed data and dynamic components. It is built on the Gymnasium framework \cite{Towers.2023} and available on GitHub\footnote{https://github.com/TimWalter/smart-energy-controller}. The main objective is minimizing the CO\textsubscript{2}eq emissions. The task is split into week-long episodes with hourly resolution, to capture daily and weekly patterns of demand and generation.
\par
The RL paradigm requires the formulation as a Markov Decision Process, consisting of a transition function, action and observation spaces, and a reward function. The transition function is given by the replayed data and the dynamics of the components in the following. The observation space is a bounded subset of $\mathbb{R}^{11+H}$, with the observables given in \cref{tab:observation_space}. The action space is summarized in  \cref{tab:action_space}. The reward function is described in \cref{ssec:reward_function}.

\begin{table}
    \caption{Observation Space.}
    \label{tab:observation_space}
    \vskip 0.15in
    \begin{center}
    \begin{small}
    \begin{sc}
    \begin{tabular}{lcr}
    \toprule
    Observable & Symbol & Replayed\\
    \midrule
    Carbon Intensity & $I$ & X\\
        Household Energy Demand & $L_{HED}$ & X\\
        Rooftop Solar Generation & $G$ & X\\
        ESS Charge & $B$ & \\
        FDR power in window & $L_{FDR}$ &\\
        Indoor temperature & $T$ & \\
        Time step & $t$ & X\\
        Month of Year & $MoY$ & X \\
        Solar Irradiation & $SI$ & X\\
        Solar Elevation & $SE$ & X\\
        Outdoor temperature & $T_a$ & X \\
        Wind Speed & $WS$ & X\\
    \bottomrule
    \end{tabular}
    \end{sc}
    \end{small}
    \end{center}
    \vskip -0.1in
    \end{table}
    
    \begin{table}
    \caption{Action Space.}
    \label{tab:action_space}
    \vskip 0.15in
    \begin{center}
    \begin{small}
    \begin{sc}
    \begin{tabular}{lcr}
    \toprule
    Action & Min. Meaning & Max. Meaning\\
    \midrule
    $a_{ess}$ & Discharge & Charge \\ 
        $a_{fdr}$ & Delay     & Expedite \\
        $a_{tcl}$ & Cooling   & Heating\\
    \bottomrule
    \end{tabular}
    \end{sc}
    \end{small}
    \end{center}
    \vskip -0.1in
    \end{table}

\subsection{Replayed Data}\label{ssec:static_components}
The data was either initially given in hourly resolution or down-sampled by averaging. The most relevant observables are displayed exemplary in \cref{fig:static_components}.
\par
The auxiliary information included weather and time data, which provided means for the agent to predict the carbon intensity of the electricity mix and the heating demand. It was sourced from the Photovoltaic Geographical Information System (PVGIS) \cite{ThomasHuld.2012}.
\par
The RSA was simulated using the PVLIB library \cite{F.Holmgren.2018}, with panel specifications from the CEC database \cite{Dobos.2012}\cite{Boyson.2007}.
\par
The household's energy demand was sourced from the UCI archive \cite{GeorgesHebrail.2006}. Since the consumption was unraveled for different rooms, the kitchen, electric water heater and air conditioner were modeled as inflexible demand, while the laundry room as flexible demand.
\par
The direct carbon intensity of the electricity mix at any given time was sourced from Electricity Maps \cite{ElectricityMaps.2023}. 
\begin{figure}[H]
    \centering
    \setlength{\abovecaptionskip}{0pt}
    \includegraphics[width=0.45\textwidth]{figures/static_components.png}
    \caption{Replayed data for the first episode.}
    \label{fig:static_components}
\end{figure}

\subsection{Energy Storage System}
The ESS is connected to the grid and the RSA. The charge is subject to the following dynamics
\begin{equation}
    B_t = \max\{0, B_{t-1} - D_s + C_t \sqrt{\nu} - \frac{D_t}{\sqrt{\nu}}\}.
\end{equation}
Here, $B_t \in [0, B_{max}]$ is the charge at time $t$ and $B_{max}$ is the capacity. Furthermore, $\nu$ denotes the round trip efficiency, $D_s$ the self-discharge rate, $C_t \in [0, C_{max}]$ the charge rate with maximum $C_{max}$, and $D_t \in [0, D_{max}]$ the discharge rate with its respective maximum $D_{max}$. 

\subsection{Flexible Demand Response}
The FDR can be influenced stochastically in a running time window of length $H$ by a signal $a_{fdr, t} \in [-1 , 1]^H$. Whether the power is consumed is determined by a Bernoulli process, with probabilities
\begin{equation}
    p_t = \text{clip}\left(s + a_{fdr, t} * \exp\left(\frac{-1}{\beta}\left|t - t_{s}\right|\right), 0, 1\right),
\end{equation}
where $s \in \{0,1\}^{H}$ indicates the desired consumption with elements $s_i = \begin{cases}
    1 & \text{if } t_i \leq t \\
    0 & \text{else}
\end{cases}$, $\beta$ is a patience parameter, and $t_{s} \in \mathbb{R}^{H}$ is the desired consumption time. If consumption is delayed more than once, $\frac{2}{H-1}$ of the power is consumed in each time step to ensure that all power has been consumed after the time window. This aims to facilitate learning by preventing infinite delays or a large correction at episode termination.

\subsection{Thermostatically Controlled Load}\label{ssec:tcl}
The TCL encapsulates all devices that aim to maintain the temperature of a given heat mass. To utilize such loads as energy storage, the temperature of the heat mass is allowed to fluctuate within a given range. The TCL is modeled as a second-order system \cite{Sonderegger.1978} with the following dynamics:
\begin{equation}
    \begin{split}
        T_t = T_{t-1} &+ \frac{1}{r_a} (T_{a, t} - T_{t-1}) \\
        &+ \frac{1}{r_b} (T_{b,t} - T_{t-1}) \\
        &+ \frac{1}{r_h} L_{TCL} a_{tcl,t} + q,
    \end{split}
\end{equation}
where $T_t$ is the indoor temperature at time $t$, $r_a$ is the thermal mass of the air, $T_{a, t}$ is the current outdoor temperature, $r_b$ is the thermal mass of the building material, and $T_{b, t}$ is the current building mass temperature, that evolves with
\begin{equation}
    T_{b, t} = T_{b, t-1} + \frac{1}{r_b} (T_{t-1} - T_{b, t-1}),
\end{equation}
$\frac{1}{r_h}$ is the power-to-heat coefficient, $L_{TCL}$ is the nominal power, 
$a_{tcl_t} \in [-1, 1]$ is the heating signal, and $q$ is the unintended heat drift. The heating signal is constrained by the desired temperature range, enforced through a backup controller as follows:
\begin{equation}
    a_{tcl, t} = \begin{cases}
        -1 & \text{if } T_t \geq T_{max} \\
        1 & \text{if } T_t \leq T_{min} \\
        a_{tcl, t} & \text{else,} 
    \end{cases}
\end{equation}
where $T_{max}$ and $T_{min}$ define the desired temperature range. 


\subsection{Reward Function} \label{ssec:reward_function}
CO\textsubscript{2}eq emissions are modeled naively as carbon intensity $I_t$ times produced energy $E_p$ minus consumed energy $E_c$. The given objective is augmented by a temperature discomfort penalty $DC$ to bias towards the desired temperature. 
The resulting reward function is
\begin{flalign}
    E_p &= G_t + D_t  && \\
    \begin{split}
        E_c &= L_t + \frac{1}{r_h} L_{TCL} a_{tcl,t} + C_t \\
        &+ \sum_{p \in U_e} p_r + \sum_{p \in U_d} \frac{2p}{H-1}
    \end{split} && \\
    DC &= \delta \exp(\left|T_t - \frac{T_{max} + T_{min}}{2}\right|) && \\
    r_t &= I_t (E_p - E_c) - DC, && 
\end{flalign}
where $U_e$ is the set of consumed FDR, $p_r$ is the remaining power after discounting, $U_d$ is the set of delayed FDR, and $\delta$ the discomfort coefficient. 
\par
Given that the task formulation is episodic, it differs from the real-world problem which has an infinite horizon. In the real-world problem, the initial state of the upcoming week is influenced by the past week. Therefore, to accurately reflect this, a correction is required at termination to evaluate the remaining state. The terminal reward is corrected by
\begin{flalign}
    reward &\mathrel{+}= I_t B_t && \\
    reward &\mathrel{+}= I_t (- \sum_{p \in U} p_r) && \\
    reward &\mathrel{+}= I_t (- r_h\left|T_t - \frac{T_{max}+T_{min}}{2}\right|) && 
\end{flalign}
where $U$ is the set of remaining FDR.

\section{Experiments}\label{sec:limitations}
\begin{table*}[t]
\label{tab:experiments}
\caption{Accumulated rewards split by origin per experiment}
\vskip 0.15in
\begin{center}
\begin{small}
\begin{sc}
\begin{tabular}{lccccr}
\toprule
Experiment & Algorithm & ESS & FDR & TCL & Discomfort \\
\midrule
\multirow{4}{*}{Full Environment} & Idle & 0.00 & 0.00 & 0.00 & 0.00 \\
& Threshold & 0.00 & 0.00 & 0.00 & 0.00 \\
& PPO & 0.00 & 0.00 & 0.00 & 0.00 \\
& SAC & 0.00 & 0.00 & 0.00 & 0.00 \\
\multirow{4}{*}{Hypothesis 1} & Idle & 0.00 & 0.00 & 0.00 & 0.00 \\
& Threshold & 0.00 & 0.00 & 0.00 & 0.00 \\
& PPO & 0.00 & 0.00 & 0.00 & 0.00 \\
& SAC & 0.00 & 0.00 & 0.00 & 0.00 \\
\multirow{4}{*}{Hypothesis 2} & Idle & 0.00 & 0.00 & 0.00 & 0.00 \\
& Threshold & 0.00 & 0.00 & 0.00 & 0.00 \\
& PPO & 0.00 & 0.00 & 0.00 & 0.00 \\
& SAC & 0.00 & 0.00 & 0.00 & 0.00 \\
\multirow{4}{*}{Hypothesis 3} & Idle & 0.00 & 0.00 & 0.00 & 0.00 \\
& Threshold & 0.00 & 0.00 & 0.00 & 0.00 \\
& PPO & 0.00 & 0.00 & 0.00 & 0.00 \\
& SAC & 0.00 & 0.00 & 0.00 & 0.00 \\
\bottomrule
\end{tabular}
\end{sc}
\end{small}
\end{center}
\vskip -0.1in
\end{table*}
To investigate the ease of the given problem initially, the first experiments were replaying only a single epsiode. Furthermore, the initial conditions, the initial indoor temperature $T_0$, building mass temperature $T_{b,0}$, and initial charge $B_0$, were deterministic for every training episode, but would be shuffled later.
\par
Since the task can be split nicely into distinct subtasks for each dynamic component, the rewards were tracked separatly to allow the analysis of performance of each subtask explicitly.
This split the reward into the following components:
\begin{flalign}
    r_{noise} &= I_t (G_t - L_t) && \\
    r_{ESS} &= I_t (D_t - C_t) && \\
    r_{FDR} &= I_t (-\sum_{p \in U_e} p_r - \sum_{p \in U_d} \frac{2p}{H-1}) && \\
    r_{TCL} &= I_t (-\frac{1}{r_h} L_{TCL} a_{tcl,t}) && \\ 
    r_{discomfort} &=- \delta exp(|T_t - \frac{T_{max} + T_{min}}{2}) &&
\end{flalign}
The reward for TCL was split, since maintaining the temperature is a very different task than minimizing the energy consumption. The result of all experiments are shown in \ref{tab:experiments}.
\par
Unfortunately, to date the task has not been solved successfully, as the RL algorithms are not outperforming the baselines. 
% Potentially policy collapse curve showing? 
Since the algorithms were widely tested and used in other domains, the problem is likely to be in the formulation of the problem. Therefore, three hypothesis are formulated to explain the poor performance of the algorithms and attempts to solve them are discussed in the following. The experiments were run with cumultating changes. The training details can be found in \ref{sec:training_procedure}.
\subsection{Hypothesis 1: Scheduling, Stochasticity, and Space Dimensionality}
The first issue investigated, was the FDR formulation. The problem of scheduling using RL is already challenging \cite{Zhang.23.10.2020} but is further complicated since the agent has only stochastic control. Furthermore, the action space and observation space dimensionality was extremely high, especially in minutely resolution, since the desired planning horizon was 24 hours. This could lead to a curse of dimensionality \cite{Sutton.2018} and was especially poor since the dimensionality would depend on the parametrization of the environment both in terms of $H$ and the resolution. 
\par
The attempted solution was to simplify the task, by allowing the agent to control the FDR deterministically and only give a scalar signal for all components. These changes are equivalent to the assumption that the optimal policy is invariant to $U_e$ as long as the sum is maintained and that $\beta = \infty$.
% TODO little bit of result talk?
\subsection{Hypothesis 2: Hard Exploration}
Another difficulty of the environment, is its vast exploration space, given by the continuity of the action and observation space, the length of the episodes, especially in minutely resolution, and the fact that the agent has to perform multiple actions per timestep. The task is furter complicated by the noise from the RSA and HED on the reward. Moreover, a lot of observables are only relevant for estimating the future carbon intensity. 
\par
To address these concerns the minutely resolution was omitted from now on. In addition, the action space for PPO was discretized into 11 bins for each component, the HED and RSA were removed from the observables and reward calculation, and the policies were trained separatly for every subtask. Moreover, observables that did not impact the reward directly besides, the timestep were removed. Lastly, the replay buffer for SAC was initialized with one episode of transitions of the threshold algorithm, providing a warm start \cite{Wang.20.06.2023}.
% TODO little bit of result talk?
\subsection{Hypothesis 3: Delayed or Deceptive reward}
Currently, the last idea to explain the poor performance of the algorithms is the reward structure for charging the ESS, expediting the FDR and heating or cooling the TCL. The rewards of those actions always occur delayed and only if the respective counter action is performed at a higher carbon intensity, which can be problematic \cite{Sutton.1984}. Moreover, the rewards might even be deceptive, since those actions are necessary for an effective policy but only ever receive negative reward.
\par
To address the time missmatch between action and reward, the reward was accumulating over time and given as an observable, however only actuated as a reward in the terminal state of the episode, which should encourage the agent to evaluate the entire trajectory of an episode together.
% TODO little bit of result talk?


\section{Discussion}\label{sec:discussion}
This paper provides an open challenge with a novel formulation of the problem of carbon intensity minimization in a smart home environment. Various experiments were run to test hypothesis on the cause of the learning inefficiency. While no single hypothesis led to a complete solution, two out of three aided the learning of some sub-tasks. A promising future direction would be to combine the various policies into an ensemble, with each controlling one subsystem. 


\bibliography{literature.bib}
\bibliographystyle{icml2023}


%%%%%%%%%%%%%%%%%%%%%%%%%%%%%%%%%%%%%%%%%%%%%%%%%%%%%%%%%%%%%%%%%%%%%%%%%%%%%%%
%%%%%%%%%%%%%%%%%%%%%%%%%%%%%%%%%%%%%%%%%%%%%%%%%%%%%%%%%%%%%%%%%%%%%%%%%%%%%%%
% APPENDIX
%%%%%%%%%%%%%%%%%%%%%%%%%%%%%%%%%%%%%%%%%%%%%%%%%%%%%%%%%%%%%%%%%%%%%%%%%%%%%%%
%%%%%%%%%%%%%%%%%%%%%%%%%%%%%%%%%%%%%%%%%%%%%%%%%%%%%%%%%%%%%%%%%%%%%%%%%%%%%%%
\newpage
\appendix
\onecolumn


\section{Further Environment Details} \label{sec:environment_details}
The replayed data was mainly recorded or simulated  in the time from 07.01.2007 to 28.12.2008. Although some episodes had to be left out due to missing data, there were more than 100 episodes, of which 95 were supposed to be used for training, 4 for testing and the rest for evaluation. The power was standardized to kilowatts, while the energy was simply kWmin or kWh depending on the resolution. However, es the time steps are equidistant, power and energy can be modeled equivalently, since the power was assumed to be constant over the time step. Only some constants had to be adapted. The location of the house was in France near Paris.
\par
The carbon intensity data was only available from 2021 and 2022 for France, so the carbon intensity might be biased, since the electricity mix might have drifted over the years. However, the daily, weekly and yearly frequencies should still be captured, although potentially drifted. The unit of carbon intensity is gram of CO\textsubscript{2} equivalent emissions per kWh or kWmin.
\par To prevent the ESS from overcharging or overdischarging, the rates were further constrained by the current charge and capacity, such that actually $C_t \in [0, \min\{C_{max}, \frac{B_{max} - B_t + D_s}{\sqrt{\nu}}\}]$ and $D_t \in [0, \min\{D_{max}, (B_t - D_S) \sqrt{\nu}\}]$.

\section{Thresholding Baseline} \label{sec:thresholding_baseline}
\begin{equation}
    a_t = \left\{
        \begin{array}{ll}
            \begin{bmatrix} \phantom{-}1 & [1]^H & \psi0.1 \end{bmatrix} & \text{if } I_t < \phi_1 \\
            \begin{bmatrix} -1 & [0]^H & \phantom{\psi}0\phantom{.1} \end{bmatrix} & \text{if } I_t > \phi_2 \\
            \begin{bmatrix} \phantom{-}0 & [0]^H & \psi0.1 \end{bmatrix} & \text{else}
        \end{array}
    \right.
\end{equation}
where $\phi_1$ and $\phi_2$ are the lower and upper threshold respectively, while $\psi = sign(T_t -\frac{T_{max} + T_{min}}{2})$.
The parameters for the thresholding algorithm were $\psi_1 = 65$ and $\psi_2 = 85$.

\section{SAC}\label{sec:sac}
SAC is a squashed Gaussian policy, that employs entropy regularization, which adds a bonus reward in each time step proportional to the current entropy of the policy in an aim to encourage exploration. Furthermore, two Q-functions are learned and their minimal estimate is used to update the policy. The objective function for SAC is:
\begin{equation}
    \max _\theta \underset{\substack{s \sim \mathcal{D} \\ \xi \sim \mathcal{N}}}{\mathbb{E}}\left[\min _{j=1,2} Q_{\phi_{\hat{j}}}\left(s, \tilde{a}_\theta(s, \xi)\right)-\alpha H(\pi_\theta \mid s)\right]
\end{equation}
where $\theta$ denotes the policy parameter, $\xi$ normal Gaussian noise, $\tilde{a}_\theta(s, \xi)$ the action sampled from the current policy, $\alpha$ the entropy regularization coefficient, and $H(\pi_\theta \mid s) = \log \pi_\theta\left(\tilde{a}_\theta(s, \xi) \mid s\right)$ the entropy of the policy. The action samples are obtained as $\tilde{a}_\theta(s, \xi) = tanh(\mu_\theta(s)+\sigma_\theta(s) \odot \xi)$


\section{PPO}\label{sec:ppo}
PPO is a trust-region based policy gradient method, which solves a constrained policy update policy via SGD. It achieves the trust region remarkably simple, by employing the following update objective:
\begin{flalign}
    \theta_{k+1} &= \arg \max _{\theta} \underset{\substack{s \sim \mathcal{D} \\ a \sim \pi_{\theta_k}}}{\mathbb{E}} \left[ L(s,a)\right]&& \\
    L(s,a)&=\min \left(\left|\frac{\pi_\theta(a \mid s)}{\pi_{\theta_k}(a \mid s)}\right|, 1\pm \epsilon \right) A^{\pi_{\theta_k}}(s, a)&&
\end{flalign}
where $A^{\pi_{\theta_k}}(s, a)$ is the advantage of the current policy, and $\epsilon$ the trust region hyperparameter, clipping the update size. Alternatively, PPO can also be implemented using a KL-divergence constraint.


\section{Training Procedure} \label{sec:training_procedure}

All experiments were run for 250 episodes and in hourly resolution. Due to the encountered difficulty all were replaying only a single episode. Furthermore, the initial conditions, the initial indoor temperature $T_0$, building mass temperature $T_{b,0}$, and initial charge $B_0$, were fixed. The training parameters are shown in \ref{tab:training_parameters}. 
\begin{table}[H]
\label{tab:training_parameters}
\caption{Training Parameters}
\vskip 0.15in
\begin{center}
\begin{small}
\begin{sc}
\begin{tabular}{lr}
\toprule
Parameter & Value\\
\midrule
$B_{max}$ & 13.5 \\
$C_{max}$ & 1 \\
$D_{max}$ & 1 \\
$\nu$ & 0.95 \\
$D_s$ & 0.01 \\
$B_0$ & 0 \\
$H$ & 25 \\
$\beta$ & 20 \\
$\delta$ & 5 \\
$T_{0}$ & 20 \\
$T_{b,0}$ & 20 \\
$r_{a}$ & 0.04 \\
$r_{b}$ & 0.1 \\
$r_{h}$ & 0.05 \\
$q$ & 0.05 \\
$L_{TCL}$ & 5 \\
$T_{max}$ & 23 \\
$T_{min}$ & 18 \\
\bottomrule
\end{tabular}
\end{sc}
\end{small}
\end{center}
\vskip -0.1in
\end{table}

\section{Split Reward}\label{sec:split_reward}
\begin{flalign}
    r_{noise} &= I_t (G_t - L_t) && \\
    r_{ESS} &= I_t (D_t - C_t) && \\
    r_{FDR} &= I_t (-\sum_{p \in U_e} p_r - \sum_{p \in U_d} \frac{2p}{H-1}) && \\
    r_{TCL} &= I_t (-\frac{1}{r_h} L_{TCL} a_{tcl,t}) && \\ 
    r_{discomfort} &=- \delta exp(|T_t - \frac{T_{max} + T_{min}}{2}) &&
\end{flalign}
The reward for TCL was split, since maintaining the temperature is a very different task than minimizing the energy consumption.
%%%%%%%%%%%%%%%%%%%%%%%%%%%%%%%%%%%%%%%%%%%%%%%%%%%%%%%%%%%%%%%%%%%%%%%%%%%%%%%
%%%%%%%%%%%%%%%%%%%%%%%%%%%%%%%%%%%%%%%%%%%%%%%%%%%%%%%%%%%%%%%%%%%%%%%%%%%%%%%


\end{document}


% This document was modified from the file originally made available by
% Pat Langley and Andrea Danyluk for ICML-2K. This version was created
% by Iain Murray in 2018, and modified by Alexandre Bouchard in
% 2019 and 2021 and by Csaba Szepesvari, Gang Niu and Sivan Sabato in 2022.
% Modified again in 2023 by Sivan Sabato and Jonathan Scarlett.
% Previous contributors include Dan Roy, Lise Getoor and Tobias
% Scheffer, which was slightly modified from the 2010 version by
% Thorsten Joachims & Johannes Fuernkranz, slightly modified from the
% 2009 version by Kiri Wagstaff and Sam Roweis's 2008 version, which is
% slightly modified from Prasad Tadepalli's 2007 version which is a
% lightly changed version of the previous year's version by Andrew
% Moore, which was in turn edited from those of Kristian Kersting and
% Codrina Lauth. Alex Smola contributed to the algorithmic style files.
