The application field of smart energy management encompasses a wide range of problems, such as heating  and cooling \cite{Blum.2021}\cite{ThomasSchreiber.2020}, flexible demand response \cite{Jin.2021} and energy storage \cite{Nakabi.2021}. Moreover, the scale and level of detail of the tackled problems vary greatly, going from single buildings \cite{Blum.2021}, to microgrids \cite{Nakabi.2021} \cite{Castellanos.04.12.2022} to eventually continent wide electricity grids \cite{Horsch.2018}. The two most promising approaches to solve such problems are classical Model Predictive Control (MPC) \cite{Basantes.2023} and Reinforcement Learning (RL) \cite{Nakabi.2021} \cite{Jin.2021} \cite{ThomasSchreiber.2020}\cite{Zhu.2022}. Furthermore, there are also hybrid approaches that combine the two methods \cite{JavierArroyo.2022}.
\\~\\
This paper is mainly inspired by \cite{Nakabi.2021}, who describe a microgrid environment with generation, storage and consumers and optimized electricity pricing
using various deep reinforcement learning algorithms.