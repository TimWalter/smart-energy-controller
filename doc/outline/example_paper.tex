\documentclass{article}

\usepackage{microtype}
\usepackage{graphicx}
\usepackage{subfigure}
\usepackage{booktabs}
\usepackage{hyperref}
\newcommand{\theHalgorithm}{\arabic{algorithm}}
\usepackage{icml2023}
\usepackage{amsmath}
\usepackage{amssymb}
\usepackage{mathtools}
\usepackage{amsthm}
\usepackage[capitalize,noabbrev]{cleveref}
\theoremstyle{plain}
\newtheorem{theorem}{Theorem}[section]
\newtheorem{proposition}[theorem]{Proposition}
\newtheorem{lemma}[theorem]{Lemma}
\newtheorem{corollary}[theorem]{Corollary}
\theoremstyle{definition}
\newtheorem{definition}[theorem]{Definition}
\newtheorem{assumption}[theorem]{Assumption}
\theoremstyle{remark}
\newtheorem{remark}[theorem]{Remark}
\usepackage[textsize=tiny]{todonotes}


% The \icmltitle you define below is probably too long as a header.
% Therefore, a short form for the running title is supplied here:
\icmltitlerunning{Outline Smart Energy Controller}

\begin{document}

\twocolumn[
\icmltitle{Outline Smart Energy Controller for \\
           Seminar on Machine learning for sequential decision making}
\icmlsetsymbol{equal}{*}

\begin{icmlauthorlist}
\icmlauthor{Tim Walter}{yyy}
\end{icmlauthorlist}

\icmlaffiliation{yyy}{Department of Scientifc Computing, Technical University of Munich, Munich, Germany}

\icmlcorrespondingauthor{Tim Walter}{tim.walter@tum.de}

\icmlkeywords{Machine Learning, ICML}

\vskip 0.3in
]
\printAffiliationsAndNotice{\icmlEqualContribution} 

\begin{abstract}
    Buildings account for a significant portion of global energy consumption and emissions. This paper explores how smart energy management systems in single-family homes can potentially reduce emissions, focusing on homes equipped with photovoltaic systems, electric batteries, and system-controllable appliances. The aim is to optimize energy consumption and generation capacities to minimize emissions, manage load flexibility, and alleviate grid pressure from fluctuating renewable energy sources. The controller also enables the sale of cleanly generated electricity to the grid at a discounted emission premium. The control mechanisms involve deep reinforcement learning algorithms benchmarked against traditional thresholding models and the theoretical optimum.
\end{abstract}

\section{Introduction}
\label{introduction}

Introduction about climate change and how our power generation has to adapt and how we can help to adapt our demand to the newly fluctuating energy generation

\section{Environment}
\label{enviroment}

The environment and reward function formulate the problem for the agent to solve by finding an optimal policy. The key components are described in the following.

\subsection{Photovoltaic System}
Simulated using the PVLIB library, providing power per timestamp ($G_t$) as part of the state.

\subsection{Battery}
The primary energy storage with charge defined by:
\begin{equation}
    B_t = B_{t-1} D_s+ C_t \sqrt{\nu} - \frac{D_t}{\sqrt{\nu}}
\end{equation}
Here, $B_t \in [0, B_{max}]$ is the charge at time $t$ and $B_{max}$ is its upper limit, these two values provide the normalized state of charge $BSoC_t = \frac{B_t}{B_{max}}$. Furthermore, $\nu$ denotes the round trip efficiency of the battery, $D_s$ is the self-discharge rate, $C_t \in [0, C_{max}]$ is the charging rate and $D_t  \in [0, D_{max}]$ is the discharging rate. The charging and discharging rates are combined into a single action for the agent, since the battery can either be charged or discharged at any given timestep. 
\begin{equation}
    a_{b, t} = \begin{cases}
            C_t, D_t = 0 & \text{if } a_{b, t} > 0 \\
            D_t, C_t = 0 & \text{else}
    \end{cases}
\end{equation}
where $a_{b, t} \in [B_t D_S \sqrt{\nu}, \frac{B_{max} - B_t D_s}{\sqrt{\nu}}]$.

\subsection{Thermostatically Controlled Load}
Under the term thermostatically controlled load (TCL) are all devices subsummed that aim to maintain the temperature of a given heat mass, such as a refrigerator, water boiler, or heat pump. To utilize such loads as energy storage, the temperature of the heat mass is allowed to fluctuate within a given range. The TCL is modeled as a second-order system
    \begin{equation}
        T_t = T_{t-1} + \frac{1}{c_a} (T_{out, t} - T_{t-1}) + \frac{1}{c_m} (T_{mass,t} - T_{t-1}) + L_{TCL} a_{tcl,t} + q
    \end{equation}
    Where $T_t$ is the measured indoor temperature, $T_{out, t}$ is the outdoor temperature, and $T_{mass,t}$ is the unobservable building mass temperature that evolves with
    \begin{equation}
        T_{mass, t} = T_{mass, t-1} + \frac{1}{c_m} (T_{t-1} - T_{mass, t-1})
    \end{equation}
    Furthermore, $c_a$ and $c_m$ are the thermal masses of the building and the air. Moreover, $q$ denotes the unintentional heating of the building, $L_{TCL}$ the nominal power of the TCL and $a_{tcl, t} \in [0, 1]$ the control decision, which is constrained by the desired temperature range, enforced through a backup controller as follows:
    \begin{equation}
        a_{tcl, t} = \begin{cases}
            0 & \text{if } T_t \geq T_{max} \\
            1 & \text{if } T_t \leq T_{min} \\
            a_{tcl, t} & \text{else} 
        \end{cases}
    \end{equation}
    where $T_{max}$ and $T_{min}$ are the maximum and minimum temperature of the desired temperature range. Another normalized state variable is the state of charge of the indoor temperature $TSoC_t = \frac{T_t - T_{min}}{T_{max} - T_{min}}$.

\subsection{Controllable Appliances}
The usage of controllable appliances can be delayed or expendited by the agent. Such appliances could be e.g. a washing machine, dishwasher, or electric vehicle. Here, the inhabitants' behaviour is modelled stochastically. The agent has access to a preplanned schedule of usage for a given timeframe, and can decide to delay or expedite certain usages by defining in each timestep whether the appliance should be used or not. Whether it is actually used is then decided by a Bernoulli process with a probability of
    \begin{equation}
        p_t^i = e^{-\frac{1}{\beta} |t - t_s|}
    \end{equation}
    with the probability mass function
    \begin{equation}
        f(a_{a, i, t}, p_t^i) = \begin{cases}
            p & \text{if } a_{a, i, t} = 1 \\
            1 - p & \text{if } a_{a, i, t} = 0
        \end{cases}
    \end{equation}
    where $p_t^i$ is the probability of executing the action $i$ at time $t$, $t_s$ is the scheduled time of usage, and $\beta$ is a patience parameter. The timeframe visible to the agent will be centered around the current timestep. If an action is not executed, while being in the visible timeframe, it is deterministically executed before exiting the timeframe. Contrary, if an action was executed and is still within the timeframe its power consumption is set to zero.

\subsection{Uncontrollable Appliances}
Uncontrollable appliances operate with a given power consumption and cannot be managed by the agent. They essentially contribute a negative generation $L_t$ at each time step. This data isn't simulated; instead, it draws from real measurements taken from a French household's consumption tracked over four years.

\subsection{Electricity Grid}
The electricity grid serves as the primary source of reward for the agent. Selling power surpluses to the grid earns an emission premium, yielding positive rewards. Conversely, if the battery or generated power falls short of meeting the load, the grid automatically balances the deficit, resulting in a negative reward tied to the current CO2eq intensity and the energy consumed.

\section{MDP Formulation}
With this environment established, a Markov Decision Process (MDP) can be defined. The problem is episodic, organized into week-long episodes, and manages controlled appliance usage within daily timeframes. Initially, the timesteps are set to one minute but might be adjusted later. There are over one hundred episodes available, spanning a two-year period.

\subsection{State}
The state can be split into three categories: the controllable state $S_c$, the uncontrollable state $S_u$, and the informal state $S_i$. The controllable state describes all parts of the environment that the agent controls either directly or indirectly. These are the state of charge for the TCL and battery and a vector $s_{a,t} \in \mathbb{R}^{N \times 2}$, that describes the scheduled usage of the controllable appliances, in terms of power consumption and desired execution time, here $N$ is the number of elements in the visible timeframe.
\begin{equation}
    s_{c, t} = [BSoC_t, TSoC_t, s_{a,t}] \in  [[0,1],[0,1],\{0,1\}^N]= S_c
\end{equation}   
The uncontrollable state describes all reward influencing parts of the environment that the agent cannot control. These are the current generation and load, as well as the CO2eq intensity for the upcoming timestep.
\begin{equation}
    s_{u, t} = [G_t, L_t, I_t] \in [\mathbb{R}^+, \mathbb{R}^+, \mathbb{R}^+]= S_u
\end{equation}
Lastly, the informal state describes all parts of the environment that are not directly influencing the reward but can be helpful for reward and state transition modelling. These are the current datetime $t$ and the current weather $W_t$.
\begin{equation}
    s_{i,t} = [t, W_t] \in S_i
\end{equation}
The state is then given by the concatenation of the three state categories.
\begin{equation}
    s_t \in S = S_c \times S_u \times S_i
\end{equation}

\subsection{Action}
The agent can influence the enviroment according to the mechanisms described in the earlier section. Consequently, the action space is given by the actions on the battery $A_b = [0, 1]$, on the TCL $A_{TCL} = [0, 1]$, the selling of generated clean energy to the grid  $A_s = [0, G_t]$, and scheduled appliances within the current timeframe $A_a = \{0, 1\}^N$. The action space is then given by the concatenation of the three action spaces.:
    \begin{equation}
        a_t \in A = A_b \times A_{TCL} \times A_s \times A_a
    \end{equation}

\subsection{Reward}
The reward function might be subject to change but initially will punish the agent mainly for emitting CO2eq and reward it for selling clean energy to the grid. The reward function is given by
    \begin{flalign}
        r_t &= c_{p} a_{s,t} - c_{e} I_t (E_c - E_p) && \\
        E_c &= L_t + a_{s, t} + a_{tcl, t} L_{TCL} + C_t + \sum_{i \in E_a}^{N} (s_{a,t})_{i, 1} && \\
        E_p &= G_t + D_t  &&
    \end{flalign}
    where $c_p$ is a coefficient determining the CO2eq premium, $c_e$ is a coefficient determining the CO2eq emission penalty, $E_c$ is the total energy consumed, $E_p$ is the total energy produced, $E_a$ is the set of scheduled appliance usages that actually got used plus the ones that were not scheduled but would leave the timeframe in the next step, and $(a_{a,t})_{i, 1}$ is the power consumption of the $i$-th usage.

\section{Algorithms}
This section will describe the algorithms used to solve the MDP. The algorithms will be benchmarked against a theoretical optimum and a thresholding model, which is also described here.
\subsection{Thresholding Model}
A quick subsection about the non machine learning baseline
\subsection{Theoretical Optimum}
A quick subsection about the theoretical optimum
\subsection{Reinforcement Learning}
Subsection that presents the reinforcement learning algorithms used to solve the MDP

\section{Results}
This section will present the results of the algorithms and compare them to the baseline and theoretical optimum.


\nocite{*}
\bibliography{literature.bib}
\bibliographystyle{icml2023}
\end{document}