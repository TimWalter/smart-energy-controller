The operations of buildings account for 30\% of global final energy consumption and 26 \%
of global energy-related greenhouse-gas emissions.\cite{IEA.06.01.2024}

Residential housing operations play a substantial role in global carbon emissions, making the optimization of energy consumption a paramount concern for environmental sustainability. This paper investigates the application of Deep Reinforcement Learning (DRL) to streamline energy usage within a single-family home, leveraging solar panels, electric batteries, heat pumps, and flexible demand mechanisms. The primary aim is to reduce the carbon footprint associated with housing operations, addressing a significant component of worldwide emissions.

Recent global data underscores the urgency of this endeavor, with housing operations contributing a notable proportion to the overall carbon footprint. This research not only explores the reduction of emissions but also emphasizes the nuanced interplay between energy consumption and grid dynamics. Aligning energy demand with periods of high renewable output, enabled by low carbon intensity electricity, not only aids environmental goals but also enhances grid stability. Dispatching reserves through batteries during periods of high carbon intensity fosters balanced power flow, contributing to a more resilient and efficient energy grid.

Moreover, an evolving landscape in electricity contracts, featuring adaptive pricing structures, introduces a potential economic incentive for inhabitants. As the energy sector increasingly transitions towards renewable sources, synchronizing residential energy usage with favorable grid conditions not only supports ecological goals but may also translate into cost savings for residents.

This paper navigates the intricacies of these interconnected factors, shedding light on the potential of DRL as a catalyst for sustainable and economically viable energy management in residential housing.